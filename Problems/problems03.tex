%% SEA CHAPTER 3 - CONCEPTUAL SYSTEM DESIGN
% SEA Question Location in \label{sea-Chapter#-Problem#}
% NEEDS UPDATING
\begin{exercises}
    \begin{exercise}
    \label{sea-03-01}
        In accomplishing a needs analysis in response to a given deficiency,what type of information would you include? Describe the process that you would use in developing the necessary information.
    \end{exercise}
    \begin{solution}
    \end{solution}
    
    \begin{exercise}
    \label{sea-03-02}
        What is the purpose of the feasibility analysis? What considerations should be addressed in the completion of such an analysis?
    \end{exercise}
    \begin{solution}
    \end{solution}
    
    \begin{exercise}
    \label{sea-03-03}
        Through a review of the literature, describe the QFD approach and how it could be applied in helping to define the requirements for a given system design.
    \end{exercise}
    \begin{solution}
    \end{solution}
    
    \begin{exercise}
    \label{sea-03-04}
        Why is the definition of system operational requirements important? What type of information is included?
    \end{exercise}
    \begin{solution}
    \end{solution}
    
    \begin{exercise}
    \label{sea-03-05}
        What specific challenges exist in defining the operational requirements for a system-of-systems (SOS) configuration? What is meant by interoperability? Provide an example.
    \end{exercise}
    \begin{solution}
    \end{solution}
    
    \begin{exercise}
    \label{sea-03-06}
        hy is it important to define specific mission scenarios (or operational profiles) within the context of the system operational requirements?
    \end{exercise}
    \begin{solution}
    \end{solution}
    
    \begin{exercise}
    \label{sea-03-07}
        What information should be included in the system maintenance concept? How is it developed (describe the steps), and at what point in the system life cycle should it be developed?
    \end{exercise}
    \begin{solution}
    \end{solution}
    
    \begin{exercise}
    \label{sea-03-08}
        How do system operational requirements influence the maintenance concept (if at all)?
    \end{exercise}
    \begin{solution}
    \end{solution}
    
    \begin{exercise}
    \label{sea-03-09}
        How does the maintenance concept affect system/product design? Give some specific examples.
    \end{exercise}
    \begin{solution}
    \end{solution}
    
    \begin{exercise}
    \label{sea-03-10}
        Select a system of your choice and develop the operational requirements for that system. Based on the results,develop the maintenance concept for the system.Construct the necessary operational and maintenance flows,identify repair policies,and apply quantitative effectiveness factors as appropriate.
    \end{exercise}
    \begin{solution}
    \end{solution}
    
    \begin{exercise}
    \label{sea-03-11}
        Refer to Figure 3.12 and assume that a similar infrastructure exists in your own community. Identify the various system capabilities (functions), illustrate (draw) an overall configuration structure (similar to that in the figure), and identify some of the critical metrics required as an input to the design of such.
    \end{exercise}
    \begin{solution}
    \end{solution}
    
    \begin{exercise}
    \label{sea-03-12}
        Refer to Figure 3.13. For a system-of-systems (SOS) configuration of your choice, describe some of the critical requirements in the design of a SOS. 
    \end{exercise}
    \begin{solution}
    \end{solution}
    
    \begin{exercise}
    \label{sea-03-13}
        In evaluating whether or not a two- or three-level maintenance concept should be specified, what factors would you consider in the evaluation process?
    \end{exercise}
    \begin{solution}
    \end{solution}
    
    \begin{exercise}
    \label{sea-03-14}
        In developing the maintenance concept, it is essential that all levels of maintenance be considered on an integrated basis. Why?
    \end{exercise}
    \begin{solution}
    \end{solution}
    
    \begin{exercise}
    \label{sea-03-15}
        Why is the development of technical performance measures (TPMs) important?
    \end{exercise}
    \begin{solution}
    \end{solution}
    
    \begin{exercise}
    \label{sea-03-16}
        Refer to Figure 3.17. Describe the steps that you would complete in developing the information included in the figure. Be specific.
    \end{exercise}
    \begin{solution}
    \end{solution}
    
    \begin{exercise}
    \label{sea-03-17}
        What is the purpose of allocation? To what depth in the system hierarchical structure should allocation be accomplished? How does it impact system design (if at all)?
    \end{exercise}
    \begin{solution}
    \end{solution}
    
    \begin{exercise}
    \label{sea-03-18}
        What is meant by functional analysis? When in the system life cycle is it accomplished? What purpose does it serve? Identify some of the benefits derived. Can a functional analysis be accomplished for any system? Can a functional analysis be accomplished for a system-of-systems (SOS) configuration?
    \end{exercise}
    \begin{solution}
    \end{solution}
    
    \begin{exercise}
    \label{sea-03-19}
        What is meant by a common function in the functional analysis? How are common functions determined? 
    \end{exercise}
    \begin{solution}
    \end{solution}
    
    \begin{exercise}
    \label{sea-03-20}
        How does the functional analysis lead into the definition of specific resource requirements in the form of hardware, software, people, data, facilities, and so on? Briefly describe the steps in the process, and include an example. What is the purpose of the block numbering shown in Figures 3.20, 3.21, and 3.22?
    \end{exercise}
    \begin{solution}
    \end{solution}
    
    \begin{exercise}
    \label{sea-03-21}
        What is the purpose of allocation? To what depth in the system hierarchical structure should allocation be accomplished? How does it impact system design (if at all)? How can allocation be applied for a SOS configuration (if at all)? 
    \end{exercise}
    \begin{solution}
    \end{solution}
    
    \begin{exercise}
    \label{sea-03-22}
        In conceptual design, there are a number of different requirements for predicting or estimating various system metrics. What approach (steps) would you apply in accomplishing such?
    \end{exercise}
    \begin{solution}
    \end{solution}
    
    \begin{exercise}
    \label{sea-03-23}
        What is the purpose of the formal design review? What are some of the benefits derived from the conduct of design reviews? Describe some of the negative aspects.
    \end{exercise}
    \begin{solution}
    \end{solution}
    
    \begin{exercise}
    \label{sea-03-24}
        What are the basic objectives in conducting a conceptual design review?
    \end{exercise}
    \begin{solution}
    \end{solution}
\end{exercises}
% SKIPPED