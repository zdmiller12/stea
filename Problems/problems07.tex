%% SEA CHAPTER 7 - ALTERNATIVES AND MODELS IN DECISION MAKING
% SEA Question Location in \label{sea-Chapter#-Problem#}
\begin{exercises}
    \begin{exercise}
    \label{sea-07-01}
        A complete and all-inclusive alternative rarely emerges in its final state. Explain.
    \end{exercise}
    \begin{solution}
    \end{solution}
    
    \begin{exercise}
    \label{sea-07-02}
        Should decision making be classified as an art or as a science?
    \end{exercise}
    \begin{solution}
    \end{solution}
    
    \begin{exercise}
    \label{sea-07-03}
        Contrast limiting and strategic factors.
    \end{exercise}
    \begin{solution}
    \end{solution}
    
    \begin{exercise}
    \label{sea-07-04}
        Discuss the various meanings of the word model.
    \end{exercise}
    \begin{solution}
    \end{solution}
    
    \begin{exercise}
    \label{sea-07-05}
        Describe briefly physical models, schematic models, and mathematical models.
    \end{exercise}
    \begin{solution}
    \end{solution}
    
    \begin{exercise}
    \label{sea-07-06}
        How do mathematical models directed to decision situations differ from those traditionally used in the physical sciences?
    \end{exercise}
    \begin{solution}
    \end{solution}
    
    \begin{exercise}
    \label{sea-07-07}
        Contrast direct and indirect experimentation.
    \end{exercise}
    \begin{solution}
    \end{solution}
    
    \begin{exercise}
    \label{sea-07-08}
        Write the general form of the evaluation function for money flow modeling and define its symbols.
    \end{exercise}
    \begin{solution}
    \end{solution}
    
    \begin{exercise}
    \label{sea-07-09}
        Identify a decision situation and indicate the variables under the control of the decision maker and those not directly under his or her control.
    \end{exercise}
    \begin{solution}
    \end{solution}
    
    \begin{exercise}
    \label{sea-07-10}
        Contrast the similarities and differences in the economic optimization functions given by Equations 7.2 and 7.3 (see Figure 7.2).
    \end{exercise}
    \begin{solution}
    \end{solution}
    
    \begin{exercise}
    \label{sea-07-11}
        Why is it not possible to formulate a model that accurately represents reality?
    \end{exercise}
    \begin{solution}
    \end{solution}
    
    \begin{exercise}
    \label{sea-07-12}
        Under what conditions may a properly formulated model become useless as an aid in decision making?
    \end{exercise}
    \begin{solution}
    \end{solution}
    
    \begin{exercise}
    \label{sea-07-13}
        Explain the nature of the cost components that should be considered in deciding how frequently to review a dynamic environment.
    \end{exercise}
    \begin{solution}
    \end{solution}
    
    \begin{exercise}
    \label{sea-07-14}
        What caution must be exercised in the use of models?
    \end{exercise}
    \begin{solution}
    \end{solution}
    
    \begin{exercise}
    \label{sea-07-15}
        Discuss several specific reasons why models are of value in decision making.
    \end{exercise}
    \begin{solution}
    \end{solution}
    
    \begin{exercise}
    \label{sea-07-16}
        Identify a multiple-criteria decision situation with which you have experience. Select the three to five most important criteria.
    \end{exercise}
    \begin{solution}
    \end{solution}
    
    \begin{exercise}
    \label{sea-07-17}
        Discuss the degree to which you think the criteria you selected in Question 16 are truly independent. Weight each criterion, check for consistency, and then normalize the weight so that the total sums to 100.Use the method of paired comparisons to rank the criteria in order of decreasing importance.
    \end{exercise}
    \begin{solution}
    \end{solution}
    
    \begin{exercise}
    \label{sea-07-18}
        Extend Question 16 to include two or three alternatives. Evaluate how well each alternative ranks against each criterion on a scale of 1 to 10. Then compute the product of the ratings and the criterion weights and sum them for each criterion to determine the weighted evaluation for each alternative.
    \end{exercise}
    \begin{solution}
    \end{solution}
    
    \begin{exercise}
    \label{sea-07-19}
        The values for three alternatives considered against four criteria are given (with higher values being better). What can you conclude using the following systematic elimination methods?
        \begin{enumerate}[label=\alph*)]
            \item Comparing the alternatives against each other (dominance).
            \item Comparing the alternatives against a standard (Rule 1 and Rule 2).
            \item Comparing criteria across alternatives (criteria ranked  ).
        \end{enumerate}
        \begin{table}[h]
        \centering
        \begin{tabular}{r r r r r r}
        \toprule
         & \multicolumn{3}{c}{\textbf{Alternative}} & \multicolumn{1}{c}{\textbf{Minimum}} \\
        \cmidrule{2-4}
        \multicolumn{1}{c}{\textbf{Criterion}} & A & B & C & \multicolumn{1}{c}{\textbf{Ideal}} & \multicolumn{1}{c}{\textbf{Standard}} \\
        \midrule
        1. & 6  & 5  & 8  & 10  & 7  \\
        2. & 90 & 80 & 75 & 100 & 70 \\
        3. & 40 & 35 & 50 & 50  & 30 \\
        4. & G  & P  & VG & E   & F  \\
        \bottomrule
        \end{tabular}
        %\caption{Table caption}
        \label{tab:sea-07-19} % Unique label used for referencing the table in-text
        %\addcontentsline{toc}{table}{Table \ref{tab:example}} % Uncomment to add the table to the table of contents
        \end{table}
    \end{exercise}
    \begin{solution}
    \end{solution}
    
    \begin{exercise}
    \label{sea-07-20}
        A small software firm is planning to offer one of four new software products and wishes to maximize profit, minimize risk, and increase market share. A weight of 65\% is assigned to annual profit potential, 20\% to profitability risk, and 15\% to market share. Use the tabular additive method for this situation and identify the product that would be best for the firm to introduce.
        \begin{table}[h]
        \centering
        \begin{tabular}{l l l l}
        \toprule
        New Product & Profit Potential & Profit Risk & Market Share \\
        \midrule
        SW \RomanNumeralCaps{1} & \$100K & \$40K & HIGH \\
        SW \RomanNumeralCaps{2} & \$140K & \$35K & MEDIUM \\
        SW \RomanNumeralCaps{3} & \$150K & \$50K & LOW \\
        SW \RomanNumeralCaps{4} & \$130K & \$45K & MEDIUM \\
        \bottomrule
        \end{tabular}
        %\caption{Table caption}
        \label{tab:sea-07-20} % Unique label used for referencing the table in-text
        %\addcontentsline{toc}{table}{Table \ref{tab:example}} % Uncomment to add the table to the table of contents
        \end{table}
    \end{exercise}
    \begin{solution}
    \end{solution}
    
    \begin{exercise}
    \label{sea-07-21}
        Convert the tabular additive results from Problem 20 into a stacked bar chart.
    \end{exercise}
    \begin{solution}
    \end{solution}
    
    \begin{exercise}
    \label{sea-07-22}
        Rework the example in Section 7.5 if the importance ratings for Better, Cheaper, Faster in Table 7.5 change from 7, 9, 4 to 9, 7, 4.
    \end{exercise}
    \begin{solution}
    \end{solution}
    
    \begin{exercise}
    \label{sea-07-23}
        Sketch a decision evaluation display that would apply to making a choice from among three automobile makes in the face of the three top criteria of your selection.
    \end{exercise}
    \begin{solution}
    \end{solution}
    
    \begin{exercise}
    \label{sea-07-24}
        Superimpose another source alternative (remanufacture) on the decision evaluation display in Section 7.5 to illustrate its expandability. Now discuss the criteria values that this source alternative would require to make it the preferred alternative.
    \end{exercise}
    \begin{solution}
    \end{solution}
    
    \begin{exercise}
    \label{sea-07-25}
        Formulate an evaluation matrix for a hypothetical decision situation of your choice.
    \end{exercise}
    \begin{solution}
    \end{solution}
    
    \begin{exercise}
    \label{sea-07-26}
        Formulate an evaluation vector for a hypothetical decision situation under assumed certainty.
    \end{exercise}
    \begin{solution}
    \end{solution}
    
    \begin{exercise}
    \label{sea-07-27}
        Develop an example to illustrate the application of paired outcomes in decision making among a number of nonquantifiable alternatives.
    \end{exercise}
    \begin{solution}
    \end{solution}
    
    \begin{exercise}
    \label{sea-07-28}
        What approaches may be used to assign probabilities to future outcomes?
    \end{exercise}
    \begin{solution}
    \end{solution}
    
    \begin{exercise}
    \label{sea-07-29}
        What is the role of dominance in decision making among alternatives?
    \end{exercise}
    \begin{solution}
    \end{solution}
    
    \begin{exercise}
    \label{sea-07-30}
        Give an example of an aspiration level in decision making.
    \end{exercise}
    \begin{solution}
    \end{solution}
    
    \begin{exercise}
    \label{sea-07-31}
        When would one follow the most probable future criterion in decision making?
    \end{exercise}
    \begin{solution}
    \end{solution}
    
    \begin{exercise}
    \label{sea-07-32}
        What drawback exists in using the most probable future criterion?
    \end{exercise}
    \begin{solution}
    \end{solution}
    
    \begin{exercise}
    \label{sea-07-33}
        How does the Laplace criterion for decision making under uncertainty actually convert the situation to decision making under risk?
    \end{exercise}
    \begin{solution}
    \end{solution}
    
    \begin{exercise}
    \label{sea-07-34}
        Discuss the maximin and the maximax rules as special cases of the Hurwicz rule.
    \end{exercise}
    \begin{solution}
    \end{solution}
    
    \begin{exercise}
    \label{sea-07-35}
        The cost of developing an internal training program for office automation is unknown but described by the following probability distribution:
        \begin{table}[h]
        \centering
        \begin{tabular}{r c}
        \toprule
        \textbf{Cost} & \textbf{Probability of Occurence} \\
        \midrule
        \$80,000  & 0.20 \\
        \$95,000  & 0.30 \\
        \$105,000 & 0.25 \\
        \$115,000 & 0.20 \\
        \$130,000 & 0.05 \\
        \bottomrule
        \end{tabular}
        %\caption{Table caption}
        \label{tab:sea-07-35} % Unique label used for referencing the table in-text
        %\addcontentsline{toc}{table}{Table \ref{tab:example}} % Uncomment to add the table to the table of contents
        \end{table}
        What is the expected cost of the course? What is the most probable cost? What is the maximum cost that will occur with a 95\% assurance?
    \end{exercise}
    \begin{solution}
    \end{solution}
    
    \begin{exercise}
    \label{sea-07-36}
        Net profit has been calculated for five investment opportunities under three possible futures. Which alternative should be selected under the most probable future criterion? Which under the expected value criterion?
        \begin{table}[h]
        \centering
        \begin{tabular}{r r r r}
        \toprule
         & \textbf{$F_1$(0.3)} & \textbf{$F_2$(0.2)} & \textbf{$F_3$(0.5)} \\
        \midrule
        $A_1$ &  \$100,000 & \$100,000 & \$380,000 \\
        $A_2$ & -\$200,000 & \$160,000 & \$590,000 \\
        $A_3$ &  \$0       & \$180,000 & \$500,000 \\
        $A_4$ &  \$110,000 & \$280,000 & \$200,000 \\
        $A_5$ &  \$400,000 & \$90,000  & \$180,000 \\
        \bottomrule
        \end{tabular}
        %\caption{Table caption}
        \label{tab:sea-07-36} % Unique label used for referencing the table in-text
        %\addcontentsline{toc}{table}{Table \ref{tab:example}} % Uncomment to add the table to the table of contents
        \end{table}
    \end{exercise}
    \begin{solution}
    \end{solution}
    
    \begin{exercise}
    \label{sea-07-37}
        Daily positive and negative payoffs are given for five alternatives and five futures in the following matrix. Which alternative should be chosen to maximize the probability of receiving a payoff of at least 9? What choice would be made by using the most probable future criterion?
        \begin{table}[h]
        \centering
        \begin{tabular}{r r r r r r}
        \toprule
         & \textbf{$F_1$(0.15)} & \textbf{$F_2$(0.20)} & \textbf{$F_3$(0.30)} & \textbf{$F_4$(0.20)} & \textbf{$F_5$(0.15)} \\
        \midrule
        $A_1$ & 12 &  8 & -4 &  0 &  9 \\
        $A_2$ & 10 &  0 &  5 & 10 & 16 \\
        $A_3$ &  6 &  5 & 10 & 15 & -4 \\
        $A_4$ &  4 & 14 & 20 &  6 & 12 \\
        $A_5$ & -8 & 22 & 12 &  4 &  9 \\
        \bottomrule
        \end{tabular}
        %\caption{Table caption}
        \label{tab:sea-07-37} % Unique label used for referencing the table in-text
        %\addcontentsline{toc}{table}{Table \ref{tab:example}} % Uncomment to add the table to the table of contents
        \end{table}
    \end{exercise}
    \begin{solution}
    \end{solution}
    
    \begin{exercise}
    \label{sea-07-38}
        The following matrix gives the payoffs in utiles (a measure of utility) for three alternatives and three possible states of nature:
        \begin{table}[h]
        \centering
        \begin{tabular}{r r r r}
        \toprule
         & \multicolumn{3}{c}{\textbf{State of Nature}} \\
         & \textbf{$S_1$} & \textbf{$S_2$} & \textbf{$S_3$} \\
        \midrule
        $A_1$ & 50 & 80 & 80 \\
        $A_2$ & 60 & 70 & 20 \\
        $A_3$ & 90 & 30 & 60 \\
        \bottomrule
        \end{tabular}
        %\caption{Table caption}
        \label{tab:sea-07-38} % Unique label used for referencing the table in-text
        %\addcontentsline{toc}{table}{Table \ref{tab:example}} % Uncomment to add the table to the table of contents
        \end{table}
        Which alternative would be chosen under the Laplace principle? The maximin rule? The maximax rule? The Hurwicz rule with  $\alpha=0.75$?
    \end{exercise}
    \begin{solution}
    \end{solution}
    
    \begin{exercise}
    \label{sea-07-39}
        The following payoff matrix indicates the costs associated with three decision options and four states of nature:
        \begin{table}[h]
        \centering
        \begin{tabular}{r r r r r}
        \toprule
         & \multicolumn{4}{c}{\textbf{State of Nature}} \\
        \textbf{Option} & \textbf{$S_1$} & \textbf{$S_2$} & \textbf{$S_3$} & \textbf{$S_4$} \\
        \midrule
        $T_1$ & 20 & 25 & 30 & 35 \\
        $T_2$ & 40 & 30 & 40 & 20 \\
        $T_3$ & 10 & 60 & 30 & 25 \\
        \bottomrule
        \end{tabular}
        %\caption{Table caption}
        \label{tab:sea-07-39} % Unique label used for referencing the table in-text
        %\addcontentsline{toc}{table}{Table \ref{tab:example}} % Uncomment to add the table to the table of contents
        \end{table}
        Select the decision option that should be selected for the maximin rule; the maximax rule; the Laplace rule; the minimax regret rule; and the Hurwicz rule with   How do the rules applied to the cost matrix differ from those that are applied to a payoff matrix of profits?
    \end{exercise}
    \begin{solution}
    \end{solution}
    
    \begin{exercise}
    \label{sea-07-40}
        A cargo aircraft is being designed to operate in different parts of the world where external navigation aids vary in quality.  The manufacturer has identified five areas where the aircraft is likely to fly and the likelihood that the aircraft will be in each area. The aircraft will be in Areas 1 through 5 with the frequency given in the table below. The table also gives the 5 navigation systems under consideration, N1, N2, N3, N4, and N5. The manufacturer has estimated the typical navigation errors to be expected in each Area and these are shown in the table. Navigation errors are in nautical miles (nm). Small navigation errors are good; large navigation errors are bad.
        \begin{table}[h]
        \centering
        \begin{tabular}{r r r r r r}
        \toprule
        \textbf{Navigation Systems} & \textbf{Area1 (0.15)} & \textbf{Area2 (0.20)} & \textbf{Area3 (0.30)} & \textbf{Area4 (0.20)} & \textbf{Area5 (0.15)} \\
        \midrule
        $N_1$ & 0.09 & 0.15 & 0.20 & 0.30 & 0.10 \\
        $N_1$ & 0.09 & 0.30 & 0.25 & 0.09 & 0.06 \\
        $N_1$ & 0.15 & 0.14 & 0.09 & 0.06 & 0.25 \\
        $N_1$ & 0.21 & 0.07 & 0.05 & 0.12 & 0.06 \\
        $N_1$ & 0.28 & 0.03 & 0.08 & 0.19 & 0.10 \\
        \bottomrule
        \end{tabular}
        %\caption{Table caption}
        \label{tab:sea-07-40} % Unique label used for referencing the table in-text
        %\addcontentsline{toc}{table}{Table \ref{tab:example}} % Uncomment to add the table to the table of contents
        \end{table}
        \begin{enumerate}[label=\alph*)]
            \item Which system(s) will maximize the probability of achieving a navigation error of 0.10 nm or less?
            \item What is the probability of achieving this navigation error?
            \item Which choice would be made by using the most probable future criterion?
        \end{enumerate}
    \end{exercise}
    \begin{solution}
    \end{solution}
    
    \begin{exercise}
    \label{sea-07-41}
        The following matrix gives the expected profit in thousands of dollars for five marketing strategies and five potential levels of sales:
        \begin{table}[h]
        \centering
        \begin{tabular}{r r r r r r}
        \toprule
         & \multicolumn{5}{c}{\textbf{Level of Sales}} \\
        \textbf{Strategy} & \textbf{$L_1$} & \textbf{$L_2$} & \textbf{$L_3$} & \textbf{$L_4$} & \textbf{$L_5$} \\
        \midrule
        $M_!$ & 10 & 20 & 30 & 40 & 50 \\
        $M_2$ & 20 & 25 & 25 & 30 & 35 \\
        $M_3$ & 50 & 40 &  5 & 15 & 20 \\
        $M_4$ & 40 & 35 & 30 & 25 & 25 \\
        $M_5$ & 10 & 20 & 25 & 30 & 20 \\
        \bottomrule
        \end{tabular}
        %\caption{Table caption}
        \label{tab:sea-07-41} % Unique label used for referencing the table in-text
        %\addcontentsline{toc}{table}{Table \ref{tab:example}} % Uncomment to add the table to the table of contents
        \end{table}
        Which marketing strategy would be chosen under the maximin rule? The maximax rule? The Hurwicz rule with $\alpha=0.4$?
    \end{exercise}
    \begin{solution}
    \end{solution}
    
    \begin{exercise}
    \label{sea-07-42}
        Graph the Hurwicz rule for all values of   using the payoff matrix of Problem 40.
    \end{exercise}
    \begin{solution}
    \end{solution}
    
    \begin{exercise}
    \label{sea-07-43}
        The following decision evaluation matrix gives the expected savings in maintenance costs (in thousands of dollars) for three policies of preventive maintenance and three levels of operation of equipment. Given the probabilities of each level of operation,   and   determine the best policy based on the most probable future criterion.
        \begin{table}[h]
        \centering
        \begin{tabular}{r r r r}
        \toprule
         & \multicolumn{3}{c}{\textbf{Level of Operation}} \\
         \textbf{Policy} & \textbf{$L_1$} & \textbf{$L_2$} & \textbf{$L_3$} \\
        \midrule
        $M_!$ & 10 & 20 & 30 \\
        $M_2$ & 22 & 26 & 26 \\
        $M_3$ & 40 & 30 & 15 \\
        \bottomrule
        \end{tabular}
        %\caption{Table caption}
        \label{tab:sea-07-43} % Unique label used for referencing the table in-text
        %\addcontentsline{toc}{table}{Table \ref{tab:example}} % Uncomment to add the table to the table of contents
        \end{table}
    \end{exercise}
    \begin{solution}
    \end{solution}
    
    \begin{exercise}
    \label{sea-07-44}
        Use the decision evaluation display of Figure 7.4 to make visible a design decision situation of your choice from Part II of this text.
    \end{exercise}
    \begin{solution}
    \end{solution}
    
    \begin{exercise}
    \label{sea-07-45}
        What should be done with those facets of a decision situation that cannot be explained by the model?
    \end{exercise}
    \begin{solution}
    \end{solution}
\end{exercises}
% SKIPPED