%% SEA CHAPTER 5 - DETAIL DESIGN AND DEVELOPMENT
% SEA Question Location in \label{sea-Chapter#-Problem#}
\begin{exercises}
    \begin{exercise}
    \label{sea-05-01}
        What are the basic differences between conceptual design, preliminary system design, and detailed design and development? Are these stages of design applicable to the acquisition of all systems? Explain.
    \end{exercise}
    \begin{solution}
    \end{solution}
    
    \begin{exercise}
    \label{sea-05-02}
        Design constitutes a team effort. Explain why. What constitutes the make-up of the design team? How can this be accomplished? How does systems engineering fit in to the process?
    \end{exercise}
    \begin{solution}
    \end{solution}
    
    \begin{exercise}
    \label{sea-05-03}
        Briefly describe the role of systems engineering in the overall design process as it is described in Chapters 3, 4, and 5.
    \end{exercise}
    \begin{solution}
    \end{solution}
    
    \begin{exercise}
    \label{sea-05-04}
        Refer to Figure 5.1. What are some of the advantages of the concurrent approach in design? Identify some of the problems that could occur in its implementation.
    \end{exercise}
    \begin{solution}
    \end{solution}
    
    \begin{exercise}
    \label{sea-05-05}
        Refer to Figures 4.4 and 4.8 (Chapter 4). As the systems engineering manager on a given program, what steps would you take to ensure that the proper integration of requirements occurs across the three life cycles (hardware, software, human) from the beginning?
    \end{exercise}
    \begin{solution}
    \end{solution}
    
    \begin{exercise}
    \label{sea-05-06}
        As a designer, one of your tasks constitutes the selection of a component to fulfill a specific design objective. What priorities would you consider in the selection process (if any)?
    \end{exercise}
    \begin{solution}
    \end{solution}
    
    \begin{exercise}
    \label{sea-05-07}
        Why are design standards (as applied to component parts and processes) important?
    \end{exercise}
    \begin{solution}
    \end{solution}
    
    \begin{exercise}
    \label{sea-05-08}
        Why are engineering documentation and the establishment of a design database necessary?
    \end{exercise}
    \begin{solution}
    \end{solution}
    
    \begin{exercise}
    \label{sea-05-09}
        Refer to Figure 5.5. When accomplishing the necessary trade-offs, there may be some confusion as to which of the three options to pursue. Describe what information is required as an input in order to evolve into a “clear-cut” approach.
    \end{exercise}
    \begin{solution}
    \end{solution}
    
    \begin{exercise}
    \label{sea-05-10}
        Describe how the application of CAD, CAM, and CAS tools can facilitate the system design process. Identify some benefits. Address some of the problems that could occur in the event of misapplication.
    \end{exercise}
    \begin{solution}
    \end{solution}
    
    \begin{exercise}
    \label{sea-05-11}
        How can CAD, CAM, and CAS tools be applied to validate the design? Provide an example of two.
    \end{exercise}
    \begin{solution}
    \end{solution}
    
    \begin{exercise}
    \label{sea-05-12}
        What is the purpose of developing a physical model of the system, or an element thereof, early in the system design process?
    \end{exercise}
    \begin{solution}
    \end{solution}
    
    \begin{exercise}
    \label{sea-05-13}
        What are some of the differences between a mock-up, an engineering model, and a prototype?
    \end{exercise}
    \begin{solution}
    \end{solution}
    
    \begin{exercise}
    \label{sea-05-14}
        Select a system (or an element of a system) of your choice and develop a design review checklist that you can use for evaluation purposes. (Refer to Figure 5.8 and Appendix B.)
    \end{exercise}
    \begin{solution}
    \end{solution}
    
    \begin{exercise}
    \label{sea-05-15}
        What are some of the benefits that can be acquired through implementation of a formal design review process?
    \end{exercise}
    \begin{solution}
    \end{solution}
    
    \begin{exercise}
    \label{sea-05-16}
        Refer to Figure 5.9. The predicted LCC value for the system, at the time of a system design review, is around $500K, which is well above the $420K design-to requirement. What steps would you take to ensure that the ultimate requirement will be met at (or before) the critical design review? Be specific.
    \end{exercise}
    \begin{solution}
    \end{solution}
    
    \begin{exercise}
    \label{sea-05-17}
        Refer to Figure 5.10. As a systems engineering manager, how would you ensure that all of the TPM requirements are being properly “tracked”?
    \end{exercise}
    \begin{solution}
    \end{solution}
    
    \begin{exercise}
    \label{sea-05-18}
        What determines whether or not a given design review has been successful?
    \end{exercise}
    \begin{solution}
    \end{solution}
    
    \begin{exercise}
    \label{sea-05-19}
        In evaluating the feasibility of an ECP, what considerations need to be addressed?
    \end{exercise}
    \begin{solution}
    \end{solution}
    
    \begin{exercise}
    \label{sea-05-20}
        Assume that an ECP has been approved by the CCB. What steps need to be taken in implementing the proposed change?
    \end{exercise}
    \begin{solution}
    \end{solution}
    
    \begin{exercise}
    \label{sea-05-21}
        What is configuration management? When can it be implemented? Why is it important?
    \end{exercise}
    \begin{solution}
    \end{solution}
    
    \begin{exercise}
    \label{sea-05-22}
        Why is baseline management so important in the implementation of the systems engineering process?
    \end{exercise}
    \begin{solution}
    \end{solution}
\end{exercises}
% SKIPPED