%% SEA CHAPTER 1 - SYSTEM SCIENCE AND ENGINEERING
% SEA Question Location in \label{sea-Chapter#-Problem#}
% ANSWERS NEED UPDATING
\begin{exercises}
    \begin{exercise}
    \label{sea-01-26}
        Describe how systems thinking differs from systems engineering.
        % What benefits could result from improving systems thinking in society?
    \end{exercise}
    \begin{solution}
        Human society is characterized by its culture. Each human culture manifests itself through the medium of technology. It takes more than a single step for society to transition from the past, to present and future technology states. A common societal response is often to make the transition and then to adopt a static pattern of behavior. A better response would be to continuously seek new but well-thought-out possibilities for advancement. Improvement in technological literacy embracing systems thinking should increase the population of individuals capable of participating in this desirable endeavor. \textbf{Reference:}
    \end{solution}
    
    \begin{exercise}
    \label{sea-01-01}
        For a system with which you are familiar, justify why it is a system according to the definition in \Cref{sec:sectors-comprising-our}.
        % Pick a system with which you are familiar and verify that it is indeed a system per the system definition given at the beginning of Section 1.1.
    \end{exercise}
    \begin{solution}
        A river system (Mississippi) is an assemblage of a watershed, tributaries, and river banks that conveys water from the continental U.S. to the Gulf of Mexico. A municipal transportation system (Chicago) is an assemblage of trains, buses, subways, etc. that transports people among many city locations. A system of organization and management (Matrix) is based on a morphology and procedure, coordinating both line and support functions. An automobile manufacturer is a combination of factories, organizations, dealerships, etc., that delivers automobiles and related support services. A home is an assemblage of land, structure, utilities, furnishings, and people that provides a supportive place to live for one or more families. \textbf{Reference:}
    \end{solution}
    
    \begin{exercise} 
    \label{sea-01-02}
        Describe the components, attributes, and relationships in the system you used in Problem \ref{sea-01-1}.
        % Name and identify the components, attributes, and relationships in the system you picked in Question 1.
    \end{exercise}
    \begin{solution}
        The major components of a home are listed in Answer 1 above. Attributes include acreage, terrain, square footage, utility capacities, styles of decorating and furnishing, personalities, and philosophies. Relationships include layout, allocation of space to people, and approaches to living together. \textbf{Reference:}
    \end{solution}
    
    \begin{exercise}
    \label{sea-01-03}
        Name any system which includes a material that transforms over the system's life cycle and identify its structural components, operating components, and flow components.
        % Pick a system that alters material and identify its structural components, operating components, and flow components.
    \end{exercise}
    \begin{solution}
        A chemica1 processing plant is composed of structural components (building, tanks, piping), operating components (pumps, valves, controls), and flow components (chemical constituents, energy, information). \textbf{Reference:}
    \end{solution}
    
    \begin{exercise} 
    \label{sea-01-04_05}
        Name any complex system and
        \begin{enumerate}[label=\alph*)]
            \item Define the hierarchy related to the system.
            \item Define the system boundaries.
        \end{enumerate}
        % Select a complex system and discuss it in terms of the hierarchy of systems.
        % Select a complex system and identify some different ways of establishing its boundaries.
    \end{exercise}
    \begin{solution}
        A dam system can be considered a complex system.
        \begin{enumerate}[label=\alph*)]
            \item todo
            \item The boundaries of a dam system can be limited to the physical dam. Alternatively, the human-modified river system, which now has a lake, can be considered a part of the dam system. The related road system, for which the dam now provides a bridge over the river, can be included. The region’s tourism service system, for which the dam system now provides an array of additional services, can be included. \textbf{Reference:}
        \end{enumerate}
    \end{solution}
    
    \begin{exercise} 
    \label{sea-01-06_07_08}
        Identify and contrast
        \begin{enumerate}[label=\alph*)]
            \item Physical versus conceptual systems.
            \item Static versus dynamic systems.
            \item Closed versus open systems.
        \end{enumerate}
        % Identify and contrast a physical and conceptual system.
        % Identify and contrast a static and a dynamic system.
        % Identify and contrast a closed and an open system.
    \end{exercise}
    \begin{solution}
        \begin{enumerate}[label=\alph*)]
            \item A physical system such as a watershed has components which manifest themselves in space and time, whereas a conceptual system such as a work breakdown structure has no physical manifestations. It is only a plan for action. Reference: Section 1.2.2 (pages 6-7).
            \item A static system such as a highway system may be contrasted with an airline system, which is a dynamic system. In the former, structure exists without activity whereas in the latter, structural components are combined with the activities of aircraft being loaded and unloaded, aircraft in flight, and controls which govern the entire operation. Reference: Section 1.2.3 (page 7).
            \item A cannon is an example of a closed system. When a cannon is fired, a one–to–one correspondence exists between the initial and final states. However, the defense contractor’s design and manufacturing organization that produced the cannon and associated projectile is an open system, with a dynamic interaction of system components. These system components must be reconfigured and adapted to cope with changing requirements. \textbf{Reference:}
        \end{enumerate}
    \end{solution}
    
    \begin{exercise} 
    \label{sea-01-15}
        For any system of the following types, name any system property
        \begin{enumerate}[label=\alph*)]
            \item Dynamic system.
            \item Steady-state system.
        \end{enumerate}
        % Give an example of a random dynamic system property and of a steady state dynamic system property.
    \end{exercise}
    \begin{solution}
        \textbf{Reference:}
    \end{solution}
    
    \begin{exercise} 
    \label{sea-01-09}
        For each of the following systems, define a unique system and describe it in terms of components, attributes, and relationships
        \begin{enumerate}[label=\alph*)]
            \item Natural system.
            \item Human-made system.
            \item Human-modified system.
        \end{enumerate}
        % Pick a natural system and describe it in terms of components, attributes, and relationships; repeat for a human-made system; repeat for a human-modified system.
    \end{exercise}
    \begin{solution}
        \begin{enumerate}[label=\alph*)]
            \item A watershed is a natural system made up of objects or components such as land, vegetation, and the watercourse; attributes such as the soil type, timber species, and the river width; and relationships such as the distribution of the attributes over the terrain. 
            \item A chemical processing plant is a human–made system with components described in Answer 3 above, attributes such as tank volume and pipe diameter, and relationships such as the flow rates and the yield of final product per energy unit utilized.
            \item A person with a pacemaker is a human-modified system with components of body parts and pacemaker parts, attributes such as body mass, diseases, attitudes, battery, controller, and electrodes, and relationships such as implantation location, rhythm, and signal strength. \textbf{Reference:}
        \end{enumerate}
    \end{solution}
    
    \begin{exercise} 
    \label{sea-01-10_11_12}
        For any Human-made system (including the system from \ref{sea-01-9})
        \begin{enumerate}[label=\alph*)]
            \item Identify the system's purpose(s) and potential metrics to present its value.
            \item Describe the system's state at any arbitrary time during operation, at least one system behavior, and an overview of the system's process.
            \item Name any two related components of the system, define the purpose of each component as it relates to the other, and the necessary attributes of the component pair such that they contribute to the purpose(s) of the entire system.
        \end{enumerate}
        % Identify the purpose(s) of the above human-made system and name some possible measures of worth.
        % For the above human-made system, describe its state at some point in time, describe one of its behaviors, and summarize its process.
        % For the above human-made system, name two components that have a relationship, identify what need each component fills for the other component, and describe how the attributes of these two components must be engineered so that the pair functions together effectively in contributing to the system’s purpose(s).
    \end{exercise}
    \begin{solution}
        \begin{enumerate}[label=\alph*)]
            \item The purposes of a chemical processing plant in a market economy are to produce one or more chemical products and possibly byproducts that can be sold at a profit while fulfilling obligations to stakeholders and the public. Measures of worth include production cost per unit volume, product quality, flexibility of product mix, benefits to stakeholders, and compatibility with society. \textbf{Reference:}
            \item During startup the state of a chemical processing plant is that pipes and vessels are filled to a certain location and empty after that location; pumps for vessels being filled are running and valves are open while other pumps are not running and valves are closed. A behavior is that when a vessel is filled, the control system turns off the pump (in a batch system) or reduces its speed (in a continuous system) and activates the next step in the process. The process is to start up, achieve the designated operational speed for each subsystem, continuously monitor the production results and make needed adjustments, and eventually shut down and clean out. \textbf{Reference:}
            \item A pump and the tank it fills have a relationship. The pump provides the material that the tank needs, while the tank provides a location where the pump can store the material it needs to deliver. The attributes of the pump must be engineered so that it can reliably move the material(s) the tank needs at an adequate rate for any given speed of overall system operation. The attributes of the tank must be engineered so that it can store the quantities of material the pump must deliver without corrosion or contamination. Thus the downstream components have the material they need to fulfill the plant’s production purpose without problems of quality or pollution. \textbf{Reference:} 
        \end{enumerate}
    \end{solution}
    
    \begin{exercise} 
    \label{sea-01-14}
        For any Human-modified system (including the system from \ref{sea-01-9}), name some positive and negative impact(s) of the modification to the natural system.
        % For a human-modified system, identify some of the ways in which the modified natural system could be degraded and some of the ways in which it could be improved.
    \end{exercise}
    \begin{solution}
        Human introduction of plant or animal species into regions where they do not naturally occur can provide the benefits of those species in the new regions, but the new species may become excessively dominant in those regions due to lack of natural enemies, crowding out or harming beneficial native species. \textbf{Reference:}
    \end{solution}
    
    \begin{exercise} 
    \label{sea-01-13}
        Give examples of each of the following
        \begin{enumerate}[label=\alph*)]
            \item First-order relationship.
            \item Second-order relationship.
            \item Redundance.
        \end{enumerate}
        % Give an example of a first-order relationship, a second-order relationship, and redundance.
    \end{exercise}
    \begin{solution}
        \begin{enumerate}[label=\alph*)]
            \item In a computer system, the power supply and system board have a first-order relationship because the system board must receive the reduced voltage produced by the power supply in order to function, and the power supply would be useless if there were no system board to perform and coordinate the computer functions. 
            \item The system board has a second-order relationship with a math coprocessor, or a video processor, or with video memory. The system board could perform the functions of these additional components, but the added components relieve the system board’s workload, thereby improving its performance.
            \item  A second power supply or a mirror image hard disk drive provide redundance, ensuring that the system board can continue receiving electrical power and the data storage function, thereby helping to assure continuation of the computer system function. \textbf{Reference:}
        \end{enumerate}
    \end{solution}
    
    \begin{exercise} 
    \label{sea-01-16}
        Name any system that operates at equilibrium and another system that degrades over time.
        % Give an example of a system that reaches equilibrium and of a system that disintegrates over time.
    \end{exercise}
    \begin{solution}
        A forest reaches equilibrium. A tree is in equilibrium until it dies, and then it disintegrates. \textbf{Reference:}
    \end{solution}
    
    \begin{exercise} 
    \label{sea-01-17}
        The United States government, for example, can be divided and described as three individual entities of the executive, legislative, and judicial branches. Create an argument for why a government of this structure should either be considered a single system or three systems.
        % Is a government with executive, legislative, and judicial branches three systems or a single system? Why?
    \end{exercise}
    \begin{solution}
        The government described is a single system because the branches thereof are functionally related. \textbf{Reference:}
    \end{solution}
    
    \begin{exercise} 
    \label{sea-01-19}
        Give any example of cybernetics and define why the example is appropriate.
        % Explain cybernetics by using an example of your choice.
    \end{exercise}
    \begin{solution}
        Cybernetics may be described and explained by considering the early mechanical version of a governor to control the revolutions per minute (RPM) of an engine. Centrifugal force, acting through a weight mechanism on the flywheel, is used to sense RPM. The outward movement of the weight against a spring acts through a link to decrease the throttle setting, thus reducing engine speed. \textbf{Reference:}
    \end{solution}
    
    \begin{exercise} 
    \label{sea-01-21}
        Do all systems at higher levels of Boulding's Hierarchy necessarily incorporate the lower levels of the hierarchy? If not, provide a specific system example.
        % Select a system at one of the higher levels in Boulding’s hierarchy and describe if it does or does not incorporate the lower levels.
    \end{exercise}
    \begin{solution}
        \textbf{Reference:}
    \end{solution}
    
    \begin{exercise} 
    \label{sea-01-22}
        Describe a novel system that may be necessary for society 50 to 100 years in the future and
        \begin{enumerate}[label=\alph*)]
            \item Define the system requirements.
            \item Define the system objectives.
        \end{enumerate}
        % Identify a societal need, define the requirements of a system that would fill that need, and define the objective(s) of that system.
    \end{exercise}
    \begin{solution}
        Efficient transportation system to the Moon and/or Mars. \textbf{Reference:}
    \end{solution}
    
    \begin{exercise} 
    \label{sea-01-25}
        For the system described in Question \ref{sea-01-22}
        \begin{enumerate}[label=\alph*)]
            \item Identify factors which led to the need for a new system.
            \item Identify other societal factors which may evolve in parallel and lead to other changes or innovations.
        \end{enumerate}
        % Name some of the factors driving technological advancement and change.
    \end{exercise}
    \begin{solution}
        \begin{enumerate}[label=\alph*)]
            \item Factors driving technological change include attempts to respond to unmet current needs and attempts to perform ongoing activities in a more efficient and effective manner, as well as social factors, political objectives, ecological concerns, and the desire for environmental sustainability. \textbf{Reference:}
            \item todo
        \end{enumerate}
    \end{solution}
    
    \begin{exercise} 
    \label{sea-01-23}
        Compare and contrast systemology and synthesis.
        % What are the similarities of systemology and synthesis?
    \end{exercise}
    \begin{solution}
        Both systemology and synthesis produce systems. Systemology produces a system of processes by which systems are brought into being and carried through the life cycle. Synthesis produces any kind of system. Synthesis is a part of systemology and also a product of systemology. \textbf{Reference:}
    \end{solution}
    
    \begin{exercise} 
    \label{sea-01-24}
        Classify a technical system.
        % What difficulty is encountered in attempting to classify technical systems?
    \end{exercise}
    \begin{solution}
        The phrase “technical system” is used to represent all types of human–made artifacts, including engineered products and processes. Classifying a technical system is generally difficult, because a technical system derives its inputs from several disciplines or fields which may be very different from one another. \textbf{Reference:}
    \end{solution}
    
    \begin{exercise} 
    \label{sea-01-27}
        Compare and contrast the attributes of the Machine (Industrial) Age and the Systems Age.
        % Identify the attributes of the Machine or Industrial Age and the Systems Age.
    \end{exercise}
    \begin{solution}
        Attributes of the Machine Age are determinism, reductionism, physical, cause and effect, and closed system thinking. The Systems Age has attributes of open systems thinking, expansionism, human–machine interfacing, automation, optimization, and goal orientation. \textbf{Reference:}
    \end{solution}
    
    \begin{exercise} 
    \label{sea-01-28}
        Identify key differences between synthetic and analytical thinking. Is one method of thinking always preferable? Why or why not?
        % Explain the difference between analytic and synthetic thinking.
    \end{exercise}
    \begin{solution}
        Analytic thinking seeks to explain the whole based on explanations of its parts. Synthetic thinking explains something in terms of its role in a larger context. \textbf{Reference:}
    \end{solution}
    
    \begin{exercise} 
    \label{sea-01-29}
        What challenges make the Systems Age unique from other periods of human evolution?
        % What are the special engineering requirements and challenges in the Systems Age?
    \end{exercise}
    \begin{solution}
        The special engineering requirements of the Systems Age are those which pertain to integration, synthesis, simulation, economic analysis, and environmental concerns, along with the necessity to bring the classical engineering disciplines to bear on the system under development through collaboration. \textbf{Reference:}
    \end{solution}
    
    \begin{exercise} 
    \label{sea-01-30}
        Compare and contrast systems engineering with other engineering disciplines.
        % What are the differences (and similarities) between systems engineering and the traditional engineering disciplines?
    \end{exercise}
    \begin{solution}
        Both systems engineering and the traditional engineering disciplines deal with technology and technical (human-made) entities. The focus of traditional engineering is on technical design of the entities in human-made systems, whereas systems engineering concentrates on what the entities are intended to do (functional design) before determining what the entities are. Traditional engineering focuses on technical performance measures, whereas systems engineering considers all requirements of the client, system owner, and/or the user group, as well as the effects on related systems. Traditional engineering focuses on designing products for their operational uses, whereas systems engineering considers all the life cycles of the systems that include its products. Traditional engineering tends to proceed from the bottom-up, whereas systems engineering favors a top-down approach. Traditional engineering favors analytic thinking while systems engineering favors synthetic thinking. Traditional engineering applies the skills of particular engineering disciplines to problems, whereas systems engineering defines problems before determining what disciplines are needed. Systems engineering provides methodologies that facilitate effective teamwork among not only the traditional engineering disciplines, but also among other technical as well as social disciplines. \textbf{Reference:}
    \end{solution}
    
    \begin{exercise} 
    \label{sea-01-32}
        Identify a system which required an interdisciplinary approach to develop \textit{or} to implement. What disciplines were required and why?
        % Give an example of a problem requiring an interdisciplinary approach and identify the needed disciplines.
    \end{exercise}
    \begin{solution}
        The problem of predicting the availability and amount of oil and natural gas from a certain geological region, which might be available to refineries and power plants in another region in future time periods, requires the disciplines of geology, petroleum engineering, regional planning, civil engineering, ecological science, transportation engineering, and economics. The validity of the prediction depends largely upon the proper utilization and interpretation of findings by the relevant disciplines and their domains of inquiry. \textbf{Reference:}
    \end{solution}
    
    \begin{exercise} 
    \label{sea-01-33}
        Identify an interdiscipline and the disciplines from which it was derived.
        % Name an interdiscipline and identify the disciplines from which it was drawn.
    \end{exercise}
    \begin{solution}
        Systems engineering is an interdiscipline (sometimes called a multidiscipline or transdiscipline) drawn mainly from the engineering disciplines, but also from mathematics, operations research, systemology, project management, and increasingly, other fields. \textbf{Reference:}
    \end{solution}
    
    \begin{exercise} 
    \label{sea-01-35_36_38}
        For the following organizations, summarize their mission statements
        \begin{enumerate}[label=\alph*)]
            \item \href{http://isss.org/}{International Society for the Systems Sciences}
            \item \href{https://www.incose.org/}{International Council on Systems Engineering}
            \item \href{https://omegalpha.org/}{Omega Alpha Association}
        \end{enumerate}
        % Go to the website for ISSS given in Section 1.7 and summarize the goal of the society.
        % Go to the website for INCOSE given in Section 1.7 and summarize the goals of the council.
        % Go to the website for OAA and compare this honor society with one that you are familiar with.
    \end{exercise}
    \begin{solution}
        \begin{enumerate}[label=\alph*)]
            \item Independent exercise. Visit http://isss.org/
            \item Independent exercise. Visit https://www.incose.org/
            \item Independent exercise. Visit https://omegalpha.org/
        \end{enumerate}
    \end{solution}
    
    \begin{exercise} 
    \label{sea-01-37}
        Explain how the goals of ISSS and INCOSE differ.
        % Contrast the goals of ISSS and INCOSE as given in Section 1.7 or on the web.
    \end{exercise}
    \begin{solution}
        Independent exercise. Refer to the solution of \ref{sea-01-35_36_38}.
    \end{solution}
\end{exercises}
% SKIPPED
% sea-01-18 Identify a system-of-systems whose analysis could yield insights not available by separately analyzing the individual systems of which it is composed.
% sea-01-20 Give a system example at any five of the levels in Boulding’s hierarchy.
% sea-01-31 Given the recommendations in Educating the Engineer of 2020, what should be added to the curriculum with which you are familiar?
% sea-01-34 Write your own (preferred) definition of systems engineering.