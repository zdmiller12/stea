%% SEA CHAPTER 8 - MODELS FOR ECONOMIC EVALUATION
% SEA Question Location in \label{sea-Chapter#-Problem#}
% ANSWERS NEED UPDATING

% only need to change the third argument in \DTLfetch{Solutions08}{ch_pr_var}{08_01_FV}{\whichBook}
\renewcommand{\V}[1]{\DTLfetch{Solutions08}{ch_pr_var}{#1}{\whichBook}}

\begin{exercises}
    \begin{exercise}
    \label{sea-8-1}
        How much money must be invested to accumulate \V{08_01_FV} in \V{08_01_periods} years at \V{08_01_i} compounded annually?
    \end{exercise}
    \begin{solution}
        \begin{equation}
            P=P/F, \V{08_01_i}, \V{08_01_periods}= \V{08_01_FV}(0.6274)= \V{08_01_PV}
        \end{equation}
    \end{solution}
    
    \begin{exercise}
    \label{sea-8-2}
        What amount will be accumulated by each of the following investments?
        \begin{enumerate}[label=\alph*)]
            \item \$8,000 at 7.2\% compounded annually over 10 years.
            \item \$52,000 at 8\% compounded annually over 5 years.
        \end{enumerate}
    \end{exercise}
    \begin{solution}
        \begin{enumerate}[label=\alph*)]
            \item $F=\$8,000(2.004)=\$16,032$
            \item $F=F/P,8,5=\$52,500(1.469)=\$77,123$
        \end{enumerate}
    \end{solution}
    
    \begin{exercise}
    \label{sea-8-3}
        What is the present equivalent amount of a year-end series of receipts of \$6,000 over 5 years at 8\% compounded annually?
    \end{exercise}
    \begin{solution}
    \end{solution}
    
    \begin{exercise}
    \label{sea-8-4}
        What is the present equivalent of a year-end series of receipts starting with a first-year base of \$1,000 and increasing by 8\% per year to year 20 with an interest rate of 8\%?
    \end{exercise}
    \begin{solution}
    \end{solution}
    
    \begin{exercise}
    \label{sea-8-5}
        What is the present equivalent of a year-end series of receipts starting with a first-year base of \$1 million and decreasing by 25\% per year to year 4 with an interest rate of 6\%?
    \end{exercise}
    \begin{solution}
    \end{solution}
    
    \begin{exercise}
    \label{sea-8-6}
        What interest rate compounded annually is involved if \$4,000 results in \$10,000 in 6 years?
    \end{exercise}
    \begin{solution}
    \end{solution}
    
    \begin{exercise}
    \label{sea-8-7}
        How many years will it take for \$4,000 to grow to \$7,000 at an interest rate of 10\% compounded annually?
    \end{exercise}
    \begin{solution}
    \end{solution}
    
    \begin{exercise}
    \label{sea-8-8}
        What interest rate is necessary for a sum of money to double itself in 8 years? What is the approximate product of $i$ and $n$ ($i$ as an integer) that establishes the doubling period? How accurate is this product of $i$ and $n$ for estimating the doubling period?
    \end{exercise}
    \begin{solution}
    \end{solution}
    
    \begin{exercise}
    \label{sea-8-9}
        An asset was purchased for \$52,000 with the anticipation that it would serve for 12 years and be worth \$6000 as scrap. After 5 years of operation, the asset was sold for \$18,000. The interest rate is 14\%.
        \begin{enumerate}[label=\alph*)]
            \item What was the anticipated annual equivalent cost of the asset?
            \item What was the actual annual equivalent cost of the asset?
        \end{enumerate}
    \end{exercise}
    \begin{solution}
    \end{solution}
    
    \begin{exercise}
    \label{sea-8-10}
        An epoxy mixer purchased for \$33,000 has an estimated salvage value of \$5,000 and an expected life of 3 years. An average of 200 pounds per month will be processed by the mixer.
        \begin{enumerate}[label=\alph*)]
            \item Calculate the annual equivalent cost of the mixer with an interest rate of 8\%.
            \item Calculate the annual equivalent cost per pound mixed with an interest rate of 12\%.
        \end{enumerate}
    \end{exercise}
    \begin{solution}
    \end{solution}
    
    \begin{exercise}
    \label{sea-8-11}
        The table below shows the receipts and disbursements for a given venture. Determine the desirability of the venture for a 14\% interest rate, based on the present equivalent comparison and the annual equivalent comparison.
        \begin{table}[h]
        \centering
        \begin{tabular}{c r r}
        \toprule
        \textbf{End of the Year} & \textbf{Receipts (\$)} & \textbf{Disbursements (\$)}\\
        \midrule
        0 & 0 & 20,000 \\
        1 & 6,000 & 0 \\
        2 & 5,000 & 4,000 \\
        3 & 5,000 & 0 \\
        4 & 12,000 & 1,000 \\
        \bottomrule
        \end{tabular}
        %\caption{Table caption}
        \label{tab:sea-8-11} % Unique label used for referencing the table in-text
        %\addcontentsline{toc}{table}{Table \ref{tab:example}} % Uncomment to add the table to the table of contents
        \end{table}
    \end{exercise}
    \begin{solution}
    \end{solution}
    
    \begin{exercise}
    \label{sea-8-12}
        A microcomputer-based controller can be installed for \$30,000 and will have a \$3,000 salvage value after 10 years and is expected to decrease energy consumption cost by \$4,000 per year.
        \begin{enumerate}[label=\alph*)]
            \item What rate of return is expected if the controller is used for 10 years?
            \item For what life will the controller give a return of 15\%?
        \end{enumerate}
    \end{exercise}
    \begin{solution}
    \end{solution}
    
    \begin{exercise}
    \label{sea-8-13}
        Transco plans on purchasing a bus for \$75,000 that will have a capacity of 40 passengers. As an alternative, a larger bus can be purchased for \$95,000 which will have a capacity of 50 passengers. The salvage value of either bus is estimated to be \$8,000 after a 10-year life. If an annual net profit of \$400 can be realized per passenger, which alternative should be recommended using a management-suggested interest rate of 15\%? Using the actual cost of money at 7.5\%?
    \end{exercise}
    \begin{solution}
    \end{solution}
    
    \begin{exercise}
    \label{sea-8-14}
        An office building and its equipment are insured to \$7,100,000. The present annual insurance premium is \$0.85 per \$100 of coverage. A sprinkler system with an estimated life of 20 years and no salvage value can be installed for \$180,000. Annual maintenance and operating cost is estimated to be \$3,600. The premium will be reduced to \$0.40 per \$100 coverage if the sprinkler system is installed.
        \begin{enumerate}[label=\alph*)]
            \item Find the rate of return if the sprinkler system is installed.
            \item With interest at 12\%, find the payout period for the sprinkler system.
        \end{enumerate}
    \end{exercise}
    \begin{solution}
    \end{solution}
    
    \begin{exercise}
    \label{sea-8-15}
        The design of a system is to be pursued from one of two available alternatives. Each alternative has a life-cycle cost associated with an expected future. The costs for the corresponding futures are given in the table below (in millions of dollars). If the probabilities of occurrence of the futures are 30\%, 50\%, and 20\%, respectively, which alternative is most desirable from an expected cost viewpoint, using an interest rate of 10\%?
        \begin{table}[h]
        \centering
        \begin{tabular}{l D{.}{.}{1} D{.}{.}{1} D{.}{.}{1} D{.}{.}{1} D{.}{.}{1} D{.}{.}{1} D{.}{.}{1} D{.}{.}{1} D{.}{.}{1} D{.}{.}{1} D{.}{.}{1} D{.}{.}{1}}
        \toprule
        \textbf{Design 1} & \multicolumn{12}{c}{\textbf{Years}} \\
        \midrule
        Future & 1 & 2 & 3 & 4 & 5 & 6 & 7 & 8 & 9 & 10 & 11 & 12 \\
        \midrule
        Optimistic & 0.4 & 0.6 & 5.0 & 7.0 & 0.8 & 0.8 & 0.8 & 0.8 & 0.8 & 0.8 & 0.8 & 0.8 \\
        Expected & 0.6 & 0.8 & 1.0 & 5.0 & 10.0 & 1.0 & 1.0 & 1.0 & 1.0 & 1.0 & 1.0 & 1.0 \\
        Pessimistic & 0.8 & 0.9 & 1.0 & 7.0 & 10.0 & 1.2 & 1.2 & 1.2 & 1.2 & 1.2 & 1.2 & 1.2 \\
        \midrule
        \textbf{Design 2} & \multicolumn{12}{c}{\textbf{Years}} \\
        \midrule
        Future & 1 & 2 & 3 & 4 & 5 & 6 & 7 & 8 & 9 & 10 & 11 & 12 \\
        \midrule
        Optimistic & 0.4 & 0.4 & 0.4 & 1.0 & 3.0 & 2.5 & 2.5 & 2.5 & 2.5 & 2.5 & 2.5 & 2.5 \\
        Expected & 0.6 & 0.8 & 1.0 & 3.0 & 6.0 & 3.0 & 3.0 & 3.0 & 3.0 & 3.0 & 3.0 & 3.0 \\
        Pessimistic & 0.6 & 0.8 & 1.0 & 5.0 & 6.0 & 3.1 & 3.1 & 3.1 & 3.1 & 3.1 & 3.1 & 3.1 \\
        \bottomrule
        \end{tabular}
        %\caption{Table caption}
        \label{tab:sea-8-15} % Unique label used for referencing the table in-text
        %\addcontentsline{toc}{table}{Table \ref{tab:example}} % Uncomment to add the table to the table of contents
        \end{table}
    \end{exercise}
    \begin{solution}
    \end{solution}
    
    \begin{exercise}
    \label{sea-8-16}
        Prepare a decision evaluation matrix for the design alternatives in \ref{sea-8-15}, and then choose the alternative that is best under the following decision rules: Laplace, maximax, maximin, and Hurwicz with $\alpha=0.6$. Assume that the choice is under uncertainty.
    \end{exercise}
    \begin{solution}
    \end{solution}
    
    \begin{exercise}
    \label{sea-8-17}
        A campus laboratory can be climate conditioned by piping chilled water from a central refrigeration plant. Two competing proposals are being considered for the piping system, as outlined in the table. On the basis of a 10-year life, find the number of hours of operation per year for which the cost of the two systems will be equal if the interest rate is 9\%.
        \begin{table}[h]
        \centering
        \begin{tabular}{l D{.}{.}{2} D{.}{.}{2}}
        \toprule
        {} & \textbf{6" System} & \textbf{8" System}\\
        \cmidrule{2-3}
        {} & \multicolumn{2}{c}{Horsepower} \\
        \cmidrule{2-3}
        Motor size & 6 & 3 \\
        \midrule
        {} & \multicolumn{2}{c}{Cost (\$)} \\
        \cmidrule{2-3}
        Pump and pipe installation & 32,000 & 44,000 \\
        Motor installation & 4,500 & 3,000 \\
        Energy per hour of operation & 3.20 & 2.00 \\
        \midrule
        Salvage value & 5,000 & 6,000 \\
        \bottomrule
        \end{tabular}
        %\caption{Table caption}
        \label{tab:sea-8-17} % Unique label used for referencing the table in-text
        %\addcontentsline{toc}{table}{Table \ref{tab:example}} % Uncomment to add the table to the table of contents
        \end{table}
    \end{exercise}
    \begin{solution}
    \end{solution}
    
    \begin{exercise}
    \label{sea-8-18}
        Replacement fence posts for a cattle ranch are currently purchased for \$4.20 each. It is estimated that equivalent posts can be cut from timber on the ranch for a variable cost of \$1.50 each, which is made up of the value of the timber plus labor cost. Annual fixed cost for required equipment is estimated to be \$1,200. If 1,000 posts will be required each year, What will be the annual saving if posts are cut?
    \end{exercise}
    \begin{solution}
    \end{solution}
    
    \begin{exercise}
    \label{sea-8-19}
        An equipment operator can buy a maintenance component from a supplier for \$960 per unit delivered. Alternatively, operator can rebuild the component for a variable cost of \$460 per unit. It is estimated that the additional fixed cost would be \$80,000 per year if the component is rebuilt. Find the number of units per year for which the cost of the two alternatives will break even.
    \end{exercise}
    \begin{solution}
    \end{solution}
    
    \begin{exercise}
    \label{sea-8-20}
        A marketing company can lease a fleet of automobiles for its sales personnel for \$35 per day plus \$0.18 per mile for each vehicle. As an alternative, the company can pay each salesperson \$0.45 per mile to use his or her own automobile. If these are the only costs to the company, how many miles per day must a salesperson drive for the two alternatives to break even?
    \end{exercise}
    \begin{solution}
    \end{solution}
    
    \begin{exercise}
    \label{sea-8-21}
        An electronics manufacturer is considering the purchase of one of two types of laser trimming devices. The sales forecast indicated that at least 8,000 units will be sold per year. Device A will increase the annual fixed cost of the plant by \$20,000 and will reduce variable cost by \$5.60 per unit. Device B will increase the annual fixed cost by \$5,000 and will reduce variable cost by \$3.60 per unit. If variable costs are now \$20 per unit produced, which device should be purchased?
    \end{exercise}
    \begin{solution}
    \end{solution}
    
    \begin{exercise}
    \label{sea-8-22}
        Machine A costs \$20,000, has zero salvage value at any time, and has an associated labor cost of \$1.15 for each piece produced on it. Machine B costs \$36,000, has zero salvage value at any time, and has an associated labor cost of \$0.90. Neither machine can be used except to produce the product described. If the interest rate is 10\% and the annual rate of production is 20,000 units, how many years will it take for the cost of the two machines to break even?
    \end{exercise}
    \begin{solution}
    \end{solution}
    
    \begin{exercise}
    \label{sea-8-23}
        An electronics manufacturer is considering two methods for producing a circuit board. The board can be hand-wired at an estimated cost of \$9.80 per unit and an annual fixed equipment cost of \$10,000. A printed equivalent can be produced using equipment costing \$180,000 with a service life of 8 years and salvage value of \$12,000. It is estimated that the labor cost will be \$3.20 per unit and that the processing equipment will cost \$4,000 per year to maintain. If the interest rate is 8\%, how many circuit boards must be produced each year for the two methods to break even?
    \end{exercise}
    \begin{solution}
    \end{solution}
    
    \begin{exercise}
    \label{sea-8-24}
        It is estimated that the annual sales of labor-saving device will be 20,000 the first year and increase by 10,000 per year until 50,000 units are sold during the fourth year. Proposal A is to purchase manufacturing equipment costing \$120,000 with an estimated salvage value of \$15,000 at the end of 4 years. Proposal B is to purchase equipment costing \$280,000 with an estimated salvage value of \$32,000 at the end of 4 years. The variable manufacturing cost per unit under proposal A is estimated to be \$8.00, but is estimated to be only \$0.26 under proposal B. If the interest rate is 9\%, which proposal should be accepted for a 4-year production period?
    \end{exercise}
    \begin{solution}
    \end{solution}
    
    \begin{exercise}
    \label{sea-8-25}
        The fixed operating cost of a machine center (capital recovery, interest, maintenance, space charges, supervision, insurance, and taxes) is $F$ dollars per year. The variable cost of operating the center (power, supplies, and other items, but excluding direct labor) is $V$ dollars per hour of operation. If $N$ is the number of hours the center is operated per year, $TC$ the annual total cost of operating the center, $TC_h$ the hourly cost of operating the center, $t$ the time in hours to process 1 unit of product, and $M$ the center cost of processing 1 unit, write expressions for the following
        \begin{enumerate}[label=\alph*)]
            \item $TC$
            \item $TC_h$
            \item $M$
        \end{enumerate}
    \end{exercise}
    \begin{solution}
    \end{solution}
    
    \begin{exercise}
    \label{sea-8-26}
        In \ref{sea-8-25}, $F=\$60,000$ per year, $t=0.2$ hour, $V=\$50$ per hour, and $N$ varies from 1,000 to 10,000 in increments of 1,000.
        \begin{enumerate}[label=\alph*)]
            \item Plot values of $M$ as a function of $N$.
            \item Write an expression for the total cost of direct labor and machine cost per unit $TC_h$ using the symbols in \ref{sea-8-25} and letting $W$ equal the hourly cost of direct labor.
        \end{enumerate}
    \end{exercise}
    \begin{solution}
    \end{solution}
    
    \begin{exercise}
    \label{sea-8-27}
        A certain firm has the capacity to produce 800,000 units per year. At present it is operating at 75\% of capacity. The income per unit is \$0.10 regardless of output. Annual fixed costs are \$28,000, and the variable cost is \$0.06 per unit. Find the annual profit or loss at this capacity and the capacity for which the firm will break even.
    \end{exercise}
    \begin{solution}
    \end{solution}
    
    \begin{exercise}
    \label{sea-8-28}
        An arc welding machine that is used for a certain joining process costs \$90,000. The machine has a life of 5 years and a salvage value of \$10,000. Maintenance, taxes, insurance, and other fixed costs amount to \$5,000 per year. The cost of power and supplies is \$28.00 per hour of operation and the total operator cost (direct and indirect) is \$65.00 per hour. If the cycle time per unit of product is 60 min and the interest rate is 8\%, calculate the cost per unit for the following unit outputs per year.
        \begin{enumerate}[label=\alph*)]
            \item 200 units
            \item 600 units
            \item 1,800 units
        \end{enumerate}
    \end{exercise}
    \begin{solution}
    \end{solution}
    
    \begin{exercise}
    \label{sea-8-29}
        A certain processing center has the capacity to assemble 650,000 units per year. At present, it is operating at 65\% of capacity. The annual income is \$416,000. Annual fixed costs are \$192,000 and the variable costs are \$0.38 per unit assembled.
        \begin{enumerate}[label=\alph*)]
            \item What is the annual profit or loss attributable to the center?
            \item At what volume of output does the center break even?
            \item What will be the profit or loss at 70\%, 80\%, and 90\% of capacity on the basis of constant income per unit and constant variable cost per unit?
        \end{enumerate}
    \end{exercise}
    \begin{solution}
    \end{solution}
    
    \begin{exercise}
    \label{sea-8-30}
        Chemco operates two plants, A and B, which produce the same product. The capacity of plant A is 60,000 gallons while that of B is 80,000 gallons. The annual fixed cost of plant A is \$2,600,000 per year and the variable cost is \$32 per gallon. The corresponding values for plant B are \$2,800,000 and \$39 per gallon. At present, plant A is being operated at 35\% of capacity and plant B is being operated at 40\% of capacity.
        \begin{enumerate}[label=\alph*)]
            \item What would be the total cost of production of plants A and B?
            \item What are the total cost and the average unit cost of the total output of both plants?
            \item What would be the total cost to the company and cost per gallon if all production were transferred to plant A?
            \item What would be the total cost to the company and cost per gallon if all production were transferred to plant B?
        \end{enumerate}
    \end{exercise}
    \begin{solution}
    \end{solution}
\end{exercises}
% SKIPPED