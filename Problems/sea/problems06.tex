%% SEA CHAPTER 6 - SYSTEM TEST, EVALUATION, AND VALIDATION
% SEA Question Location in \label{sea-Chapter#-Problem#}
\begin{exercises}
    \begin{exercise}
    \label{sea-6-1}
        How are the specific requirements for “system test, evaluation, and validation” determined?
    \end{exercise}
    \begin{solution}
    \end{solution}
    
    \begin{exercise}
    \label{sea-6-2}
        Describe the “system test, evaluation, and validation” process. What are the objectives?
    \end{exercise}
    \begin{solution}
    \end{solution}
    
    \begin{exercise}
    \label{sea-6-3}
        Describe the basic categories of test and their application. How do they fit into the total system evaluation process?
    \end{exercise}
    \begin{solution}
    \end{solution}
    
    \begin{exercise}
    \label{sea-6-4}
        Refer to Figure 2.4. How is “evaluation” accomplished in each of the phases shown (i.e., Blocks 0.6, 1.5, 2.3, 3.4, and 4.4)?
    \end{exercise}
    \begin{solution}
    \end{solution}
    
    \begin{exercise}
    \label{sea-6-5}
        What special test requirements should be established for systems operating within a higher-level system-of-systems (SOS) structure? Identify the steps required to ensure compatibility with the other systems in the SOS configuration. 
    \end{exercise}
    \begin{solution}
    \end{solution}
    
    \begin{exercise}
    \label{sea-6-6}
        When should the planning for “system test, evaluation, and validation” commence? Why? What information is included?
    \end{exercise}
    \begin{solution}
    \end{solution}
    
    \begin{exercise}
    \label{sea-6-7}
        Select a system of your choice and develop a system test and evaluation plan (i.e., a TEMP, or equivalent).
    \end{exercise}
    \begin{solution}
    \end{solution}
    
    \begin{exercise}
    \label{sea-6-8}
        Why is it important to establish the proper level of logistics and maintenance support in conducting a system test? What would likely happen in the absence of adequate test procedures? What would likely happen in the event that the assigned test personnel were inadequately trained? What would likely happen in the absence of the proper test and support equipment? Be specific.
    \end{exercise}
    \begin{solution}
    \end{solution}
    
    \begin{exercise}
    \label{sea-6-9}
        Why is it important to develop and implement a good data collection, analysis, reporting, and feedback capability? What would likely happen in the absence of such?
    \end{exercise}
    \begin{solution}
    \end{solution}
    
    \begin{exercise}
    \label{sea-6-10}
        Describe some of the objectives in the development of a data collection, analysis, and reporting capability.
    \end{exercise}
    \begin{solution}
    \end{solution}
    
    \begin{exercise}
    \label{sea-6-11}
        What data are required to measure the following: cost effectiveness, system effectiveness, operational availability, life-cycle cost, reliability, and maintainability?
    \end{exercise}
    \begin{solution}
    \end{solution}
    
    \begin{exercise}
    \label{sea-6-12}
        How would you evaluate the adequacy of the system’s supporting supply chain?  What measures are required?
    \end{exercise}
    \begin{solution}
    \end{solution}
    
    \begin{exercise}
    \label{sea-6-13}
        How would you measure and evaluate system sustainability?
    \end{exercise}
    \begin{solution}
    \end{solution}
    
    \begin{exercise}
    \label{sea-6-14}
        In the event that the system evaluation process indicates a “noncompliance” with a specific system requirement, what steps should be taken to correct the situation?
    \end{exercise}
    \begin{solution}
    \end{solution}
    
    \begin{exercise}
    \label{sea-6-15}
        In the event that a system design change is being recommended to correct an identified design deficiency, what should be done in response?
    \end{exercise}
    \begin{solution}
    \end{solution}
    
    \begin{exercise}
    \label{sea-6-16}
        Why is configuration management so important in the implementation of the “system test, evaluation, and validation” process?
    \end{exercise}
    \begin{solution}
    \end{solution}
    
    \begin{exercise}
    \label{sea-6-17}
        Why is the “feedback loop” important (assuming that it is important)?
    \end{exercise}
    \begin{solution}
    \end{solution}
    
    \begin{exercise}
    \label{sea-6-18}
        Briefly describe the customer/producer/supplier activities (and interrelationships) in system evaluation.
    \end{exercise}
    \begin{solution}
    \end{solution}
    
    \begin{exercise}
    \label{sea-6-19}
        What benefits can be derived from good test and evaluation reporting?
    \end{exercise}
    \begin{solution}
    \end{solution}
    
    \begin{exercise}
    \label{sea-6-20}
        Describe the role of systems engineering in the “system test, evaluation, and validation” process.
    \end{exercise}
    \begin{solution}
    \end{solution}
\end{exercises}
% SKIPPED